\section{UdeMy Cmake}

\section{Project Software Toolkit}

\begin{verbatim}
Doxygen - create html documentation based on your codebase
Conan/VCPKG Packaging - How to install and use external libraries
Unit Testing
Code Coverage
CI Testing -- use all these tools in continuous integration, Github actions


We'll see how to create an html documentation based on your codebase
\end{verbatim}

\section{Installation Commands}

\begin{verbatim}
sudo apt-get update
sudo apt-get upgrade

# Mandatory
sudo apt-get install gcc g++ gdb
sudo apt-get install make cmake
sudo apt-get install git
sudo apt-get install doxygen
sudo apt-get install python3 python3-pip

# Optional
sudo apt-get install lcov gcovr
sudo apt-get install ccache
sudo apt-get install cppcheck
sudo apt-get install llvm clang-format clang-tidy
sudo apt-get install curl zip unzip tar

# for VSCODE

in extension, download franneck94 c/c++ extension pack
And the coding tools extension pack
Then, with command palette (in view), Config: generate C config file. It configures all tools the same as the
teacher's
\end{verbatim}


\section{Command Line Options}


\begin{verbatim}

\end{verbatim}
# CMake Tutorial

## Generating a Project

```bash
cmake [<options>] -S <path-to-source> -B <path-to-build>
```

Assuming that a CMakeLists.txt is in the root directory, you can generate a project like the following.

```bash
mkdir build
cd build
cmake -S .. -B . # Option 1
cmake .. # Option 2
```

Assuming that you have already built the CMake project, you can update the generated project.

```bash
cd build
cmake .
```

\subsection{Generator for GCC and Clang}

\begin{verbatim}
cd build
cmake -S .. -B . -G "Unix Makefiles" # Option 1
cmake .. -G "Unix Makefiles" # Option 2
\end{verbatim}


\subsection{Generator for MSVC}

\begin{verbatim}
cd build
cmake -S .. -B . -G "Visual Studio 16 2019" # Option 1
cmake .. -G "Visual Studio 16 2019" # Option 2
\end{verbatim}


\subsection{Specify the Build Type}


Per default, the standard type is in most cases the debug type.
If you want to generate the project, for example, in release mode you have to set the build type.

\begin{verbatim}
cd build
cmake -DCMAKE_BUILD_TYPE=Release ..
\end{verbatim}

\subsection{Passing Options}

If you have set some options in the CMakeLists, you can pass values in the command line.

\begin{verbatim}
cd build
cmake -DMY_OPTION=[ON|OFF] ..
\end{verbatim}


\subsection{Specify the Build Target (Option 1)}


The standard build command would build all created targets within the CMakeLists.
If you want to build a specific target, you can do so.

\begin{verbatim}
cd build
cmake --build . --target ExternalLibraries_Executable
\end{verbatim}

The target *ExternalLibraries\_Executable* is just an example of a possible target name.
Note: All dependent targets will be built beforehand.

\subsection{Specify the Build Target (Option 2)}


Besides setting the target within the cmake build command, you could also run the previously generated Makefile (from the generating step).
If you want to build the *ExternalLibraries\_Executable*, you could do the following.

\begin{verbatim}
cd build
make ExternalLibraries_Executable
\end{verbatim}


\section{Run the Executable}

After generating the project and building a specific target you might want to run the executable.
In the default case, the executable is stored in *build/5\_ExternalLibraries/app/ExternalLibraries\_Executable*, assuming that you are building the project *5_ExternalLibraries* and the main file of the executable is in the *app* dir.

\begin{verbatim}
cd build
./bin/ExternalLibraries_Executable
\end{verbatim}


\subsection{Different Linking Types}

\begin{verbatim}
add_library(A ...)
add_library(B ...)
add_library(C ...)
\end{verbatim}

\subsection{Public}

\begin{verbatim}
target_link_libraries(A PUBLIC B)
target_link_libraries(C PUBLIC A)
\end{verbatim}

When A links in B as *PUBLIC*, it says that A uses B in its implementation, and B is also used in A's public API. Hence, C can use B since it is part of the public API of A.

\subsection{Private}

\begin{verbatim}
target_link_libraries(A PRIVATE B)
target_link_libraries(C PRIVATE A)
\end{verbatim}


When A links in B as *PRIVATE*, it is saying that A uses B in its
implementation, but B is not used in any part of A's public API. Any code
that makes calls into A would not need to refer directly to anything from
B.

\subsection{Interface}

\begin{verbatim}
add_library(D INTERFACE)
target_include_directories(D INTERFACE {CMAKE_CURRENT_SOURCE_DIR}/include)
\end{verbatim}

In general, used for header-only libraries.

## Different Library Types

### Library

A binary file that contains information about code.
A library cannot be executed on its own.
An application utilizes a library.

### Shared

- Linux: *.so
- MacOS: *.dylib
- Windows: *.dll

Shared libraries reduce the amount of code that is duplicated in each program that makes use of the library, keeping the binaries small.
Shared libraries will however have a small additional cost for the execution.
In general the shared library is in the same directory as the executable.

### Static

- Linux/MacOS: *.a
- Windows: *.lib

Static libraries increase the overall size of the binary, but it means that you don't need to carry along a copy of the library that is being used.
As the code is connected at compile time there are not any additional run-time loading costs.
