\chapter{Engineering}
\section{Builds}

\subsection{Build Configuration (build target)}

\begin{verbatim}
    Project settings determining how your project will be built. 
    Build configuration includes executable name, project arch and library files. 
    It specifies keepings or strippings of debugging info, compiler optimization details 
\end{verbatim}

\subsection{Build Release Configuration}

\begin{verbatim}
    Optimized for size and performance, no debugging information.
    With all optimization, now testing for code performance.

    When the Hello World program (from lesson 0.7) was built using Visual Studio,
    Executable produced in the debug configuration was 65kb, 
    Executable built in the release version was 12kb. 
    The difference is largely due to the extra debugging information kept in the debug build.
\end{verbatim}


\subsection{Build Testing Configuration}

\begin{verbatim}
    Debug configuations turns off all optimizations, includes debugging information,
    Jason Turner would say "with as much information as possible"
    Such configs makes your programs larger and slower, but much easier to debug. 
\end{verbatim}

\subsection{G++ Builds}
\begin{verbatim}
GCC / Clang? 
    -ggdb  // cmd line debugging
           // This is the GNU Debugger !?
    -ggdb O2 -DNDEBUG for release builds. ??
    -g++ O2 -DNDEBUG for release builds. ??
\end{verbatim}

\section{Debugging}

\begin{verbatim}
    Debug configuations turns off all optimizations, includes debugging information,
    Jason Turner would say "with as much information as possible"
    Such configs makes your programs larger and slower, but much easier to debug. 
\end{verbatim}

\section{GDB}
\section{Testing Framework Catch2}

\section{Benchmarking}
\section{Threads}


