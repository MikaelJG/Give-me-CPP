\chapter{C++ Basics}

\section{Variables}

\begin{verbatim}
    int: integers                   // 4 bytes
    double: floating-point numbers  // double 8 bytes
    char: individual characters     // 1 byte
    float:                          // 4 bytes
    long:                           // 4 or 8 bytes (platform dependent)
    long long:                      // 8 bytes
    bool: true/false                // 1 byte

    // use sizeof for size

std::cout << "Size of int: " << sizeof(numInt); 
\end{verbatim}

\section{Auto}

\begin{verbatim}
constexpr double calculate_pi() {
    return 22/7;
}

constexpr auto pi = calculate_pi();

int main()
{
    const auto radius = 1.5;
    const auto area = pi * radius * radius;
    std::cout << area;

}

Automatic deduction of value types. Standard c++11

constexpr auto calculate_pi() { // Return type deduction 
                                // for normal functions.
    return 22/7;
}

constexpr auto pi = calculate_pi();

int main()
{
    const auto radius = 1.5;
    const auto area = pi * radius * radius;
    std::cout << area;
}

// const auto
const auto result = std::count( /*stuff */);

or, if you prefer:
// auto const

auto const result = std::count( /*stuff*/ );
\end{verbatim}

\subsection{Auto Expensive Conversion}

Avoid potential expensive conversions. Same as ranged-for loops. 

Auto requires initialization, same as const. 

\begin{verbatim}
const std::string value = get_string_value();

// What is the return type of get_string_value()? 

If its std::string_view or const char *,
we'll get costly conversion on all compilers with no diagnostic. 

// avoids conversion
const auto value = get_string_value();

Plus, auto return types simplifies code
\end{verbatim}

\section{Constexp}

Compile-time generation of code and data. Compile-time constant.
`constexpr` is used for compile-time evaluation.
It ensures that an expression or function can be computed at compile time.
It reduces runtime overhead.
The program avoids the need to perform those computations at runtime. 

The computations are already done during the compilation process, 
The program doesn't incur the additional time and resources required
to perform them during runtime execution.

\begin{verbatim}
const double pi = 3.141593;

int main()
{
   const double radius = 1.5;
   const double area = pi * radius * radius;
   std::cout << area;
}

constexpr double pi = 3.141593;

int main()
{
   const double radius = 1.5;
   const double area = pi * radius * radius;
   std::cout << area;
}

// OR EVEN, Generate it at compile-time.

constexpr double calculate_pi() {
    return 22/7;
}

constexpr double pi = calculate_pi();

int main()
{
   const double radius = 1.5;
   const double area = pi * radius * radius;
   std::cout << area;
}

// This function can be executed at compile-time
constexpr double calculate_pi() {
    return 22/7;
}

// This value will be available at compile-time 
constexpr double pi = calculate_pi();
\end{verbatim}

\subsection{Overcomplicated Constexpr}

\begin{verbatim}

// static const data known at compile time. 
static const std::vector<int> angles{-90,-45,0,45,90};
\end{verbatim}

\subsection{Useful Constexpr}

\begin{verbatim}
// Moving static const to static constexpr.
static constexpr std::array<int, 5> angles{-90,45,0,45,90};

Here static constexpr makes sure the object
is not reinitialized each time the function is encountered.

With static (see object lifetimes), the variable lasts for the lifetime of the program.
It will be initialized only once. 

    the size of the array is now known at compile time
    We've removed dynamic allocations
    We no longer pay the cost of accessing a static

    " when you see a const, always ask yourself: "is this value known at compile time? 
    If it is, what would it take to make the value constexpr?" 
\end{verbatim}

\section{Const Char}

\begin{verbatim}
#include <filesystem>
namespace fs = std::filesystem;

int main() {
    fs::path path = "your_directory_path";

    for (const auto &entry : fs::directory_iterator(path)) {
        const char *cstr = entry.path().c_str();
        std::cout << cstr << '\n';
    }
    return 0; // return success
}
\end{verbatim}

\section{Synthax}
\subsection{Ternary Operator}
\begin{verbatim}
std::cout << (coin == 0 ? "Heads" : "Tails") << "\n";
\end{verbatim}

\subsection{Relational Operators}

\begin{verbatim}
    ==              equal to
    !=              not equal to
    >               greater than
    <               less than
    >=              greater than or equal to
    <=              less than or equal to
\end{verbatim}

\subsection{Logical Operators}

\begin{verbatim}
    &&              and 
    ||              or
    !               not

    if (hunger && anger){}

    ( !true )       not false
    ( !(10 < 11) )  not condition 
\end{verbatim}

\subsection{String Manipulation}

\begin{verbatim}
reversed_text += text

std::string chapOnePath = parentDirectory + "/chap_1";
\end{verbatim}

\subsection{Scope}

\begin{verbatim}
run() {
    {
        Is this in scope? //This variable is out of scope.?
    {
    
}
\end{verbatim}

\subsection{Chaining}

\begin{verbatim}
int age = 28;
std::cout << age << "years old.\n";

template <typename T>
struct MathOperation {
    T value;

    MathOperation(T val) : value(val) {}

    template <typename U>
    MathOperation<U> add(U val) {
        return MathOperation<U>(value + val);
    }

    template <typename U>
    MathOperation<U> multiply(U val) {
        return MathOperation<U>(value * val);
    }

    void print() {
        std::cout << "Result: " << value << std::endl;
    }
};

int main() {
    MathOperation<int> operation(5);
    operation.add(3).multiply(2).print();

    return 0; // return success
}
\end{verbatim}

\subsection{Sstream}

`std::stringstream` is a C++ class. It handling string-based input and output operations,
allowing you to read from and write to strings as if they were input/output streams.

\begin{verbatim}
#include <map>
#include <sstream>

int main() {
    std::string text = "lorem ipsum dolor sit amet, consectetur adipiscing elit.";

    // convert text to lowercase
    for (char& c : text) {
        c = std::tolower(c);
    }

    std::map<std::string, int> wordfreq;
    std::istringstream iss(text);
    std::string word;

    // count the frequency of each word in the text
    while (iss >> word) {
        wordfreq[word]++;
    }

    // display the word frequencies
    for (const auto& pair : wordfreq) {
        std::cout << pair.first << ": " << pair.second << std::endl;
    }

    return 0; // return success
}
\end{verbatim}


\section{Touples}


\section{Enums}

\begin{verbatim}
enum class Day {
    Monday,
    Tuesday,
    Wednesday,
};

int main() {
    Day today = Day::Tuesday;

    if (today == Day::Saturday || today == Day::Wednesday) {
    } else {}

    return 0; // return success
}

enum Color {
    Red,
    Green,
    Blue
};

void printColor(Color color) {
    switch (color) {
        case Red:
            break;
        case Green: // ...
            break;
        case Blue: // ...
            break;
    }
}

int main() {
    Color favoriteColor = Color::Green;
    printColor(favoriteColor);

    return 0; // return success
}
\end{verbatim}

\subsection{Enum Array Mapping}

\begin{verbatim}
enum class Fruit {
    Apple,
    Banana,
    Orange
};

const std::array<std::string, 3> fruitNames = {
    "Apple",
    "Banana",
    "Orange"
}

const std::string fruitNames[] = { // c-style array
                                   // size by initializer
    "Apple",
    "Banana",
    "Orange"
};

int main() {
    Fruit selectedFruit = Fruit::Banana;
    int fruitIndex = static_cast<int>(selectedFruit);

    std::cout << "Selected fruit: " << fruitNames[fruitIndex] << std::endl;

    return 0; // return success
}
\end{verbatim}

\subsection{Enum Vector Mapping}

\begin{verbatim}
enum class Month {
    January,
    February,
    March // ...
};

const std::vector<std::string> monthNames = {
    "January",
    "February",
    "March" // ... 
};

int main() {
    Month currentMonth = Month::May;
    int monthIndex = static_cast<int>(currentMonth);

    std::cout << "Current month: " << monthNames[monthIndex] << std::endl;

    return 0; // return success
}
\end{verbatim}


\section{Arrays}

A fixed-size stack-based container. Having the size type information gives more optimization oppotunities.

\begin{verbatim}
#include <array> // c++ 11
    std::array<char, 128> second = {'H', 'e', 'l'} // from library
                            // fixed size of 128
                            // has .begin(), .end(), .at(), .size() 

    sint arr[] = {1, 2, 3}; // c-style array 
                            // size determined by initializer's list
                            // fixed at compile-time 

std::array<Type, Size> data

#include <numeric>
#include <array>

template<typename Value_Type>
std::array<Value_Type, 3> get_data(const Value_Type &v1, const Value_type &v2,
                                   const Value_type &v3)
{
    std::array<Value_Type 3> data;
    data[0] = v1;
    data[1] = v2;
    data[2] = v3;
    return data;
}

// no dynamic allocation, 
// win-win scenario with knowing the size of the data struture at compile time.

template<typename> VT> // takes 3 parameters
std::array<VT, 3> get_data(const VT &v1, const VT &v2, const VT &v3)
{
    return {v1, v2, v3};
}

template<typename> VT> // takes 4 parameters
std::array<VT, 4> get_data(const VT &v1, const VT &v2, const VT &v3, const VT &v4))
{
    return {v1, v2, v3, v4};
}

... 
// If only there was a way to avoid all this code duplication !!!
\end{verbatim}

\subsection{Dynamic Array Allocation}

Achieved using pointers and dynamic memory allocation operators, such as `new` and `delete`. 
While arrays are considered static containers,
dynamic arrays allow you to allocate memory at runtime.

\begin{verbatim}
int size = 5; // desired size of the array
int* dynamicArray = new int[size]; // allocate memory for the array

// Access and modify elements of the dynamic array
dynamicArray[0] = 10;
dynamicArray[1] = 20;
// ...

// Deallocate the memory when it's no longer needed
delete[] dynamicArray;

#include <array>

int main() {
    std::array<int, 3> ar{1,2,3};

    int* dyn_ar = new int[4];

    dyn_ar[0] = 10;
    dyn_ar[1] = 20;
    dyn_ar[2] = 30;
    dyn_ar[3] = 40;
    dyn_ar[4] = 50;
    dyn_ar[5] = 60;
    dyn_ar[6] = 70;

    for (int i = 0; i < 7; i++) {
        std::cout << dyn_ar[i] << " ";
    }
    std::cout << std::endl; // prints 10, 20, 30, 40.. 70.

    delete[] dyn_ar;

    return 0; // return success
}

\end{verbatim}

`new int[size]` dynamically allocates memory. 
`delete[] dynamicArray` deallocates the memory to avoid memory leaks.

Alternatively, using smart pointers or container classes like `std::vector` can help automate memory management
and provide safer alternatives for dynamic arrays.

\section{Vectors}

Vectors are \textbf{dynamic array-like container that can grow or shrink.}

\begin{verbatim}
  std::vector<double> subway_adult; // value is 0.0 is default
  std::vector<double> location(2); // initialize two elements! 
}

std::vector<char> vowels = {'a', 'e', 'i', 'o', 'u'};
std::vector vec{1,2,3}; // now possible! 

int main(int argc, char* argv[]) {
    std::vector<std::string> arguments(argv + 1, argv + argc);
}

template<typename Value_Type>
std::vector<Value_Type> get_data(const Value_Type &v1, const Value_Type &v2,
                                 const Value_Type &v3)

{
    std::vector<Value_Type> data;
    data.push_back(v1);
    data.push_back(v2);
    data.push_back(v3);
    return data;
}
\end{verbatim}

\section{Size\_t}

\begin{verbatim}
template<typename Value_Type>
struct Data {
    Data(const std::size_t size)
      : data(new Value_Type[size]) // constructor
    {
    }

    ~Data() { delete [] data; }

    Value_Type *data;
};
\end{verbatim}

In the `Data` struct, the constructor takes a `std::size\_t` parameter named `size`,
which specifies the desired size of the `data` array. 
By using `std::size\_t` as the parameter type, 
it ensures that the value provided for `size` is appropriate for representing the size of the array.

Inside the constructor, the `data` member is \textbf{allocated dynamically using `new`}.

The size of the array is specified as the value of `size`, which is of type `std::size\_t`. 
This ensures that the correct amount of memory is allocated for the array based on the given size.

\section{Iterators}
\subsection{Conditionals}

\begin{verbatim}
if (coin == 0) {
} else {}
}
\end{verbatim}

\subsection{Switch Statements}

\begin{verbatim}
int main() {
  int number = 9;
  switch(number) {
    case 1 : // ...
      std::cout << "case one";
      break;
    case 2 :
      break;
    default : // ...
      break;
  }
}
\end{verbatim}

\subsection{Loops}

\begin{verbatim}
while (guess != 8) {
  std::cout << "Wrong guess, try again: ";
  std::cin >> guess;
}

for (int i = 0; i < 20; i++) {}  // incrementing
for (int i = 20; i > 0; i--) {}  // decrementing
\end{verbatim}

\section{Ranges}

\begin{verbatim}
#include <format>
#include <string_view>

void print_map(const auto &map,
               const std::string_view &key_desc = "key",
               const std::string_view &value_desc = "value")
{
    const auto print_key_value = [&](const auto &data) { 
        const auto &[key, value] = data;
        std::puts(std::format("{}: '{}' {}: '{}'",
                         key_desc, key, value_desc, value).c_str());
    };

    for_each(map, print_key_value);
}

#include <ranges>

int main()
{
    std::vector<int> ints{1, 2, 3, 4, 5};
    auto even = [](int i){ return 0 == i % 2; };
    auto square = [](int i){ return i * i; };
                                              
    for (int i : ints | std::view::filter(even) | std::view::transform(square)) {
        std::cout << i << ' ';
    }
}
\end{verbatim}

\subsection{Ranges Pipes}

I are operators for composing ranges in C++20.

They chain range adaptors, transforming or filtering operations.
Pipes take the output of one range and passes it as the input to the next range adaptor,
allowing you to compose multiple operations on a range in a concise and readable way.

\begin{verbatim}
auto even = [](int i){ return 0 == i % 2; };
auto square = [](int i){ return i * i; };

for (int i : ints | std::view::filter(even) | std::view::transform(square)) {
    std::cout << i << ' ';
}

`ints`: The input range of integers.
`std::view::filter(even)`: Filters the `ints` range, keeping only the even numbers.
`std::view::transform(square)`: Transforms the filtered range by squaring each element.
`int i : ...`: Iterates over the resulting transformed range and assigns each element to `i`.
`std::cout << i << ' ';`: Prints each element `i` separated by a space.
\end{verbatim}

\subsection{Range-Based for Loops}


Iterate over container elements (array, vector or std::list). It doesn't work for forward\_list.


Works with anything that has begin()
and end() members/functions, C-Style arrays and initializer lists.

\begin{verbatim}

for (const auto &element : container) {}

std::string str = "Hello";
for (char character : str) {
    std::cout << character << '\n';
}

template<typename Map>
void print_map(const Map &map, const std::string &key_desc = "key",
                               const std::string &value_desc = "value")
{
    for (const auto &data : map)
    {
        std::cout << key_desc << ": '" << data_itr->first << "' "
                  << value_desc << ": '" << data_itr->second << "'\n";
    }
}

for (const auto &value : container) {} // for each element in the container

Use clang-tidy's modernize-loop-convert check.

template<typename Map>
void print_map(const Map &map, const std::string &key_desc = "key",
                               const std::string &value_desc = "value")
{
    for (const auto &data : map)
    {
        std::cout << key_desc << ": '" << data.first << "' "
                  << value_desc << ": '" << data.second << "'\n";
    }
}
\end{verbatim}

\subsection{Accidental Conversions}

\begin{verbatim}
for (const int value : container_of_double) {
    // accidental conversion, possible warning
}
\end{verbatim}

\subsection{Accidental Slicing}

\begin{verbatim}
for (const base value : container_of_derived) {
    // accidental silent slicing
}

If container_of_derived holds objects of a derived class. 
Base is the base class.
The loop is iterating over the container and assigning each
derived object to a base object (value) due to object slicing.

Object slicing occurs because the base object can only store
the base class's attributes and behavior. Additional defined 
class attributes will be lost during the assignment or copy.

To avoid accidental slicing, you use pointers or references.

// no problem
for (const auto &value : container) {
    // no possible accidental conversion
}

Using pointers or references ensures that 
the derived objects retain their specific attributes and behavior.

const auto & for non-mutating loops
auto & for mutating loops
auto && 
// only when you have to with weird types like std::vector<bool>,
// or if moving elements out of the container
\end{verbatim}

\section{Functions}

\begin{verbatim}
void eat() {
  std::cout << "nom nom\n";
}

bool even(int num) {
  return ( num % 2 == 0 ? true : false );
  // this should be tested
}
\end{verbatim}

\subsection{Inline Functions}

The compiler inserts the function’s body on the function call.

\begin{verbatim}
inline 
void eat() {
  std::cout << "nom nom\n";
}

-- single line in header file is slower --
void Cookie::eat() {std::cout << "nom nom\n";}

-- goodnight.cpp --
std::string goodnight1(std::string thing1) {
  return "Goodnight, " + thing1 + ".\n";
}
-- goodnight.h --
std::string goodnight1(std::string thing1);

-- night.hpp -- 
inline
std::string goodnight1(std::string thing1) {
  return "Goodnight, " + thing1 + ".\n";
}
\end{verbatim}

\subsection{Class Functions (Member Functions)}

Functions inside of classes.

\begin{verbatim}
class Musician {
private:
    int instruments;

public:
    int getMyVariable() const {
        return myVariable;
    }

    void setMyVariable(int newValue) {
        myVariable = newValue;
    }
};

int main() {
    MyClass obj;
    obj.setMyVariable(42);

    int value = obj.getMyVariable();
    return 0;
}
\end{verbatim}

\subsection{Functions Public}

\begin{verbatim}
class City {
  int population; 
 
public:         // accessible outside of the class
  void add_resident() { 
    population++;
  }

private:        // private to this class
  bool is_capital;
};

-- song.h --
class Song {
  std::string title;
  std::string artist;

public:
  void add_title(std::string new_title);
  std::string get_title();
  
  void add_artist(std::string new_artist);
  std::string get_artist();
};

-- song.cpp --
#include "song.hpp"

void Song::add_title(std::string new_title) {
  title = new_title;
}

std::string Song::get_title() {
  return title;
}

void Song::add_artist(std::string new_artist) {
  artist = new_artist;
}

std::string Song::get_artist() {
  return artist;
}

-- city.h -- 
#include "city.hpp"
 
class City {
  std::string name;
  int population;
 
public:
  City(std::string new_name, int new_pop);
 
};

-- city.cpp --
City::City(std::string new_name, int new_pop)
  : name(new_name), population(new_pop) {} 

-- main.cpp --
City ankara("Ankara", 5445000);

-- song.h -- 
class Song {
  std::string title;
  std::string artist;

public:
  Song(std::string new_title, std::string new_artist);

  void add_title(std::string new_title);
  std::string get_title();

  void add_artist(std::string new_artist);
  std::string get_artist();
};

-- song.cpp --
#include "song.hpp"

Song::Song(std::string new_title, std::string new_artist)
  : title(new_title), artist(new_artist) {}

void Song::add_title(std::string new_title) {
  title = new_title;
}

std::string Song::get_title() {
  return title;
}

void Song::add_artist(std::string new_artist) {
  artist = new_artist;
}

std::string Song::get_artist() {
  return artist;
}

-- song.h --

class Song {
  std::string title;
  std::string artist;

public:
  Song(std::string new_title, std::string new_artist);

  std::string get_title();
  std::string get_artist();
};

-- song.cpp --
#include "song.hpp"

Song::Song(std::string new_title, std::string new_artist)
  : title(new_title), artist(new_artist) {}

std::string Song::get_title() {
  return title;
}

std::string Song::get_artist() {
  return artist;
}

-- main.cpp --
#include "song.hpp"

int main() { 
  Song back_to_black("Back to Black", "Amy Winehouse");

  std::cout << back_to_black.get_title() << "\n";
  std::cout << back_to_black.get_artist() << "\n";
}
\end{verbatim}

\subsection{Functions Overloading}

Accepts many types as parameters.

Change behavior based on parameter's type.

\begin{verbatim}
// one must be true
    Each has different type parameters.
    Each has a different number of parameters.
    
-- num.cpp --
int fancy_number(int num1, int num2) {
  return num1 - num2 + (num1 * num2);
}

int fancy_number(int num1, int num2, int num3) {
  return num1 - num2 - num3 + (num1 * num2 * num3);
} // different number of params.

int fancy_number(double num1, double num2) {
  return num1 - num2 + (num1 * num2);
}               // different type of params.

-- num.hpp --
int fancy_number(int num1, int num2);
int fancy_number(int num1, int num2, int num3);
int fancy_number(double num1, double num2);
\end{verbatim}

\section{Parameters}

\begin{verbatim}
void get_emergency_number(std::string emergency_number) {}

struct ComplexType {
    int value;
    std::vector<int> data;
};

void processComplexType(const ComplexType& complexParam) {
    std::cout << "Value: " << complexParam.value << std::endl;
    std::cout << "Data:";
    for (int num : complexParam.data) {
        std::cout << " " << num;
    }
    std::cout << std::endl;
}

int main() {
    ComplexType complexObj;
    complexObj.value = 42;
    complexObj.data = {1, 2, 3, 4, 5};

    processComplexType(complexObj);

    return 0;
}

Many Parameters

double get_tip(double price, double tip, bool total_included) {
    get_tip(0.25, true, 45.50); // will not work. Order matters. 
}

void name_x_times(std::string name, int x){
  while (x > 0) {
    std::cout << name << "\n";
    x--;
  }
}
\end{verbatim}

\subsection{Pass-By-Value}

Passing by value makes modification impossible, since the passed variable is out of scope.
Pass by reference to modify the passed parameter.

\begin{verbatim}
void useless_function(int val) {
    val = val * 2;
}
\end{verbatim}


\subsection{Pass-By-Reference}

Modify the argument's value.
Avoid making copies of a variable/object for performance reasons.

\begin{verbatim}
void swap_num(int &i, int &j) {
  int temp = i;
  i = j;
  j = temp;
}
 
int main() {
  int a = 100;
  int b = 200;
 
  swap_num(a, b);
}

int triple(int &i) {
  i = i * 3;
  return i;
}

int main() {
  int num = 1;
  std::cout << triple(num) << "\n";
}

template<typename T>
void modifyVector(std::vector<T>& vec) {
    for (auto& element : vec) {
        element *= 2;
    }
}

template<typename T>
void printVector(const std::vector<T>& vec) {
    for (const auto& element : vec) {
        std::cout << element << " ";
    }
    std::cout << std::endl;
}

int main() {
    std::vector<int> numbers = {1, 2, 3, 4, 5};

    printVector(numbers);  // Output: Original vector: 1 2 3 4 5
    modifyVector(numbers);

    printVector(numbers);  // Output: Modified vector: 2 4 6 8 10
    return 0;
}
\end{verbatim}

\subsection{Parameters Default Values}

\begin{verbatim}
struct Chapters {
    bool IsToPrint = true;
};

struct Chapters {
    bool IsToPrint;

    Chapters() : IsToPrint(true) {} // with default constructor
};

int main() {
    Chapters c1;
    Chapters c2 = {false}; //  list initialization
                           // works when only one bool in struct

    // c1.IsToPrint will be true
    // c2.IsToPrint will be false

    return 0;
}
\end{verbatim}

\subsection{Default Arguments}

\begin{verbatim}
void intro(std::string name, std::string lang = "C++");
\end{verbatim}

\section{References}

\subsection{References Variables}

\textbf{A second name for an existing variable}.

\begin{verbatim}
int &sonny = songqiao; // Sonny, a reference to songqiao;

\\ Changes to the reference happens to the original.
\\ Aliases cannot be changed to alias something else.

  int soda = 99;
  int &pop = soda;
  pop++; // soda and pop equal at 100

Using a reference (&) instead of making a copy of the elements is more efficient,
especially when dealing with large objects or containers. 

By using a reference, the loop avoids creating a new copy of each element
, reducing unnecessary memory usage and improving performance.

struct ComplexStruct {
    int value1;
    int value2;
};

void modifyStruct(ComplexStruct& ref) {
    ref.value1 = 100;
    ref.value2 = 200;
}

int main() {
    int x = 10;
    int& ref1 = x;              // Reference variable ref1 refers to x
    ref1 = 20;                  // Modifying ref1 will also modify x

    ComplexStruct obj;
    obj.value1 = 50;
    obj.value2 = 60;
    ComplexStruct& ref2 = obj;  // Reference variable ref2 refers to obj

    modifyStruct(ref2);         // Modifying ref2 will also modify obj
    return 0;
}
\end{verbatim}

\subsection{Dereference}

Obtain the pointer's pointed value. 

\begin{verbatim}
Declaration? * creates a pointer.
Not a declaration? * is a dereference operator.

int main() {
  int power = 9000;
  int* ptr = &power;
  std::cout << *ptr; // prints 9000
}
\end{verbatim}

\subsection{Pass-By-Reference}

Now, when called, the function can modify the argument's value.
Avoid making copies of a variable/object for performance reasons.

\begin{verbatim}
//Calling swap_num(), variables a and b's value will be modified
//because they are passed by reference

void swap_num(int &i, int &j) {
  int temp = i;
  i = j;
  j = temp;
}
 
int main() {
  int a = 100;
  int b = 200;
 
  swap_num(a, b);
 
  std::cout << "A is " << a << "\n";
  std::cout << "B is " << b << "\n";
}

WHEN? To MODIFY the arguments’ values.

int triple(int &i) {
  i = i * 3;
  
  return i;
}

int main() {
  int num = 1;
  std::cout << triple(num) << "\n";
}

template<typename T>
void modifyVector(std::vector<T>& vec) {
    for (auto& element : vec) {
        element *= 2;
    }
}

template<typename T>
void printVector(const std::vector<T>& vec) {
    for (const auto& element : vec) {
        std::cout << element << " ";
    }
    std::cout << std::endl;
}

int main() {
    std::vector<int> numbers = {1, 2, 3, 4, 5};

    std::cout << "Original vector: ";
    printVector(numbers);  // Output: Original vector: 1 2 3 4 5

    modifyVector(numbers);

    std::cout << "Modified vector: ";
    printVector(numbers);  // Output: Modified vector: 2 4 6 8 10

    return 0;
}
\end{verbatim}

\subsection{References Const}

\begin{verbatim}
int triple(const int &i) { // save computational cost
  return i * 3;            // don't make a copy of the argument
}

int square(const int &i) {
  return i * i;
}

int main() {
  int side = 5;
  std::cout << square(side) << "\n";
}
\end{verbatim}

\subsection{Reference Operator}

\begin{verbatim}
int soda = 99; 
int &pop = soda;
pop++;
\end{verbatim}

\subsection{Address Operator}

\begin{verbatim}
int porcupine = 3; 
std::cout << &porcupine << "\n"; 

// 0x7ffd7caa5b54
\end{verbatim}

\chapter{Memory Allocation}
\section{Pointers}

\begin{verbatim}
Declaration? * creates a pointer.
Not a declaration? * is a dereference operator.

int main() {
  int power = 9000;
  int* ptr = &power;
  std::cout << *ptr; // prints 9000
}

int* number;         // Declaring
double* decimal;
char* character;

int gum = 8;
int* ptr = &gum;     // int* [declare a pointer]
                     // ptr [pointer's name]
                     // &gum [store gum's memory address in ptr]

int* number;         // Declaration standard
int *number;
int * number;        // All Synthaxically valid

\end{verbatim}

\subsection{Null Pointers}

Ptr to a yet unknown memory address.

\begin{verbatim}
int* ptr; // declared, not yet initialized
\end{verbatim}

\subsection{Nullptr}

Type safe pointer since C++14, I think. A replacement for NULL.

\begin{verbatim}
int* ptr = nullptr;
\end{verbatim}

\subsection{Smart Pointers}

\subsection{Pointer of Pointer}

\begin{verbatim}
int main() {
    int value = 5;
    int* ptr = &value;
    int** ptrToPtr = &ptr;

    return 0;
}
\end{verbatim}

\subsection{Allocation}

\subsection{Deallocation}

\subsection{Freeing Memory}
\subsection{Leaking Memory}

\section{Classes}

\begin{verbatim}
A C++ class is a user-defined type.

class City {
}; // needs semicolon
\end{verbatim}

\subsection{Class Components}

\begin{verbatim}
Class components are called class members
Attributes and methods are class members

class City {
  int population; // attribute
 
public:
  void add_resident() { // method
    population++;
  }
};
\end{verbatim}


\section{Span}

A non-owning, lightweight and flexible view over a contiguous sequence of elements.
A `span` is a pointer to the first element and the number of elements in the sequence.
A non-owning view, \textbf{it can be used to work with the array without copying its elements}.

Pass and manipulate data ranges, such as sub-arrays or portions of containers,
without memory allocation overhead or ownership.

It allows you to write generic functions that operate on different containers
without requiring specific container types.

\begin{verbatim}
#include <span>

void printSpan(std::span<int> sp) {
    for (int element : sp) {}
}

int main() {
    std::array<int,5> arr{1, 2, 3, 4, 5};
    std::span<int> span(arr, 5);

    printSpan(span); // takes span and prints elements
    return 0;
}
\end{verbatim}

\section{Templates}

The ultimate in the DRY principle. Write a template that has types and values filled in at compile-time.

\begin{verbatim}
template <typename meaningful_Type>
meaningful_Type get_smallest(meaningful_Type num1, meaningful_Type num2) {
  return num2 < num1? num2 : num1;
}

int main() {
  std::cout << get_smallest(100, 60) << "\n";
  std::cout << get_smallest(2543.2, 3254.3) << "\n";
}

template types are generated by the compiler at compile time
Do not need any kind of type-erasure (like Java generics do)

Highly efficient runtime code possible, 
as good as (or better than) hand writing the various options
template system is Turing complete (not necessarily a good thing)

    template<typename SomeType>
    struct S { // struct can do anything it wants with this type
    };

// declare a class template that can hold anything we want
template<typename Value_Type>
struct Data {
    Data(const std::size_t size)
      : data(new Value_Type[size])
    {
    }

    ~Data() { delete [] data; }

    Value_Type *data;
};

// declare a function template that take 3 params of the same type
// and passes that type on to the 'Data' template

template<typename Value_Type>
Data<Value_Type> get_data(const Value_Type &v1, const Value_Type &v2,
                          const Value_type &v3)
{
    {Data<Value_Type> d(3);
    d.data[0] = v1; d.data[1] = v2; d.data[2] = v3;
    return d;
}
\end{verbatim}

\subsection{Templates Basics}

\begin{verbatim}
template<typename T>
T divide(T numerator, T denominator) {
    return numerator / denominator;
}
\end{verbatim}

\subsection{Variadic Templates}

Drastic simplification of code needing to match a variable number of parameters.

\begin{verbatim}
#include <array>

// require at least one parameter and it sets the type
template<typename VT, typename ... Params>
std::array<VT, sizeof...(Params)+1> get_data(const VT &v1, const Params& ...params)
{
    return {v1, params...};
}


template<typename VT, Typename ... P> // variadic template
std::array<VT, sizeof...(P)> get_data(const P & ... params) // param expansion
{
    return {params...}; // pack expansion
}

template<typename> VT>
std::array<VT, 3> get_data(const VT &v1, const VT &v2, const VT &v3) // 3 params
{
    std::array<VT, 3> data;
    data[0] = v1; data[1] = v2; data[2] = v3;
    return data;
}

template<typename> VT>
std::array<VT, 2> get_data(const VT &v1, const VT &v2) // 2 params
{
    std::array<VT, 2> data;
    data[0] = v1; data[1] = v2;
    return data;
}

template<typename> VT>
std::array<VT, 1> get_data(const VT &v1) // 1 params
{
    std::array<VT, 1> data;
    data[0] = v1;
    return data;
}

// This bothers.
template<typename Value_Type>
struct Data {
    Data(const std::size_t size)
      : data(new Value_Type[size])
    {
    }

    ~Data() { delete [] data; }

    Value_Type *data;
};
\end{verbatim}

\subsection{CTAD - Class Template Argument Deduction}

\begin{verbatim}
template<typename VT, typename ... Params>
auto get_data(const VT &v1, const Params & ... params)
{
    return std::array{v1, params...}; // auto deduced size/type
}

std::vector vec{1,2,3}; // now possible! 
and now we can simplify the template arguments

/include <array>

template<typename ... Params>
auto get_data(const Params & ... params)
{
    return std::array{params...};
}
\end{verbatim}

\subsection{Templates Auto Return Types}

\begin{verbatim}
template<typename Numerator, typename Denominator>
auto divide(Numerator numerator, Denominator denominator)
{
    return numerator / denominator;
}
\end{verbatim}


\section{Structs}

The only difference between a struct and a class is public properties. With
structs, all properties are public by default.

\begin{verbatim}
struct Double_Data {
    Double_Data(const std::size_t size) : data(new double[size] { }
    ~Double_Data() { delete [] data; }
    double *data;
   };

// I want one for ints!
struct Int_Data {
    Int_Data(const std::size_t size) : data(new int[size] { }
    ~Int_Data() { delete [] data; }
    int *data;
   };

// I want one for floats!
struct Float_Data {
    Float_Data(const std::size_t size) : data(new float[size] { }
    ~Float_Data() { delete [] data; }
    float *data;
   };

// If only there was some way to avoid repeating ourselves here!!
\end{verbatim}

\subsection{Structs Against Classes}
\begin{verbatim}
The only difference between them is that struct has all members by default public.
Using struct makes examples shorter and easier to read. 
\end{verbatim}

\section{Lambdas}

Unnamed function objects occasionally capturing.

\begin{verbatim}
auto lambda = [/*captures*/](int param1){ return param1 * 10; };

std::vector<int> ints{1, 2, 3, 4, 5};
auto even = [](int i){ return 0 == i % 2; };
auto square = [](int i){ return i * i; }; 

for (int i : ints | std::view::filter(even) | std::view::transform(square)) {
    std::cout << i << ' ';
}
\end{verbatim}

\subsection{Lambdas Captures}

Access variables from its surrounding scope.
Used outside variable within their body.

The captures section can be left empty if
the lambda does not need to access any external variables.

\begin{verbatim}
auto lambda = [/*captures*/](int param1){ return param1 * 10; };
\end{verbatim}

\subsection{Lambdas Captures by Values}

A copy of the variable is made.

\begin{verbatim}
auto lambda = [param](int param1){ return param1 * 10; };
\end{verbatim}

\subsection{Lambdas Captures References}

\begin{verbatim}
auto lambda = [&param](int param1){ return param1 * 10; };
\end{verbatim}

\subsection{Lambdas Captures by Values and References}

Capture multiple variables, 

\begin{verbatim}
int x = 42;
double y = 3.14;

auto lambda = [x, &y](int param1){ return x * param1 + y; };

    // x is captured by value, creating a copy
    // y is captured by reference, with access to original
\end{verbatim}

\subsection{Lambda Auto}
\begin{verbatim}
template<typename Map>
void print_map(const Map &map, const std::string key_desc = "key",
                               const std::string value_desc = "value")
{
    for_each(begin(map), end(map),
        [&](auto &data) { /// Sick !!
                          /// This makes a generic lambda.
            std::cout << key_desc << ": '" << data.first << "' "
                      << value_desc << ": '" << data.second << "'\n";
            }
    );
}
\end{verbatim}

\subsection{Lambdas Generic Lambdas}

Create implicit templates by simply using the auto keyword.

\begin{verbatim}
auto lambda = [/*captures*/](auto ... params){
    return std::vector<int>{params...};
}

standard c++14
\end{verbatim}

\subsection{Lambdas Variadic}

\subsection{Lambdas Const Object Initionalization}

\begin{verbatim}
const auto data = [](){ // no parameters
    std::vector,int. result;
    // fill result with things. 
    return result;
}(); // immediately invoked

Because of RVO, using a lambda will not add overhead and may increase performance.
What is RVO?

Yet, you don't want to make class members constl.
It can break things silently. 
\end{verbatim}

\section{Collections}

\begin{verbatim}
double *get_data() {
    double *data = new double[3];

    data[0] = 1.1;
    data[1] = 2.2;
    data[2] = 3.3;

    return data;
}

struct Double_Data {
    Double_Data(const std::size_t size) : data(new double[size] {} // allocate
    ~Double_Data() { delete [] data; }
    double *data;
   };

Double_Data get_data() {
    Double_Data data(3);
    data.data[0] = 1.1; data.data[1] = 2.2; data.data[2] = 3.3;
    return data;
   };

double sum_data(const Double_Data &d) {
    return d.data[0] + d.data[1] + d.data[2];
}

int main() {
    return sum_data(get_data()); // no leak, but we'll come back to it
}

#include <vector>
#include <map>

struct Person {
    std::string name;
    int age;
    std::vector<std::string> hobbies;
};

int main() {
    std::vector<Person> people = {
        {"Alice", 25, {"Reading", "Painting"}},
        {"Bob", 30, {"Gaming", "Hiking"}},
        {"Charlie", 20, {"Cooking", "Photography"}}
    };

    std::map<std::string, Person> personMap;
    for (const auto& person : people) {
        personMap[person.name] = person;
    }

    for (const auto& pair : personMap) {
        std::cout << "Name: " << pair.first << std::endl;
        std::cout << "Age: " << pair.second.age << std::endl;
        std::cout << "Hobbies: ";
        for (const auto& hobby : pair.second.hobbies) {
            std::cout << hobby << " ";
        }
        std::cout << std::endl;
    }

    return 0;
}
\end{verbatim}


\section{Concepts}

I want to have two versions of a function, one takes a Floating Point, the other an Integral value.
Allows us to specify the requirements for a type, implicitly creating a template
that constrains how a function can be used. Standard: C++20

\begin{verbatim}
old c++ implementation, using type\_traits.

#include <type_traits>

template<typename T,
         typename std::enable_if<std::is_floating_point<T>::value, int>::type = 0>
auto func(T f) -> decltype(f * 3) { return f * 3; }

template<typename T,
         typename std::enable_if<std::is_integral<T>::value, int>::type = 0>
auto func(T i) -> decltype(i + 3) { return i + 3; }

c++17 implementation, using type\_traits still.
#include <type_traits>

template<typename T,
         std::enable_if_t<std::is_floating_point_v<T>, int> = 0>
auto func(T f) { return f * 3; }

template<typename T,
         std::enable_if_t<std::is_integral_v<T>, int> = 0>
auto func(T i) { return i + 3; }

C++20 implementation using concepts, 

#include <concepts>

auto func(std::floating_point auto f) { return f * 3; }
auto func(std::integral auto i) { return i + 3; }
\end{verbatim}

\subsection{Auto Concept}

\begin{verbatim}
// C++20's auto concept or further constrained to something that
// has values that can be destructured into 2 elements.

void print_map(const auto &map, const std::string &key_desc = "key",
                                const std::string &value_desc = "value")
{
    for (const auto &[key, value] : map) /// strucuted binding
    {
        std::cout << key_desc << ": '" << key << "' "
                  << value_desc << ": '" << value << "'\n";
    }
}

// implicitely created a template for us. Just like our lamdas did!
\end{verbatim}

\section{Constants}

The most important tool to write clean code.
An object declared const or accessed via a const referece or const pointer cannot be modified.
It forces us to think about initialization and lifetime of objects, which affects performance. 

Plus, it communicates meaning to readers. 
If a variable is not const, ask why not?
Would using a lambda or adding a named function allow you to make the value const?

\begin{verbatim}
int triple(int const i) { // we know parameters won't change
  return i * 3;
}

const double pi = 3.141593; // tells the compiler the value can't change

int main()
{
    const double radius = 1.5;
    const double area = pi * radius * radius;
    std::cout << area;

    // east const or west const, the same
    const int i = 5;
    int const j = 6;
}

Const everything that's not constexpr

Is this a good const?

double *get_data() {
    double *data = new double[3];
    data[0] = 1.1; data[1] = 2.2; data[2] = 3.3;
    return data;
}

double *sum_data(double *data) {
    return data[0] + data[1] + data[2]; // uncaught leak
}

int main() {
    return sum_data(get_data());
}

// If only there was some way to automatically delete things when they are no longer needed...
\end{verbatim}


\section{Memory Allocation}

\begin{verbatim}
// In declaration & is a reference

Memory Address (&)

int porcupine = 3;
std::cout << &porcupine << "\n"; // Memory address
                                 // 0x7ffd7caa5b54

Dynamic Memory Allocation

int main() {
    int size = 5;
    int* dynamicArray = new int[size];

    for (int i = 0; i < size; ++i) {
        dynamicArray[i] = i + 1;
    }

    for (int i = 0; i < size; ++i) {
        std::cout << dynamicArray[i] << " ";
    }

    delete[] dynamicArray;
    return 0;
}
\end{verbatim}

\section{Factories}

Design pattern.

\begin{verbatim}
struct S {
    S() = default;
    S(S&&) = delete;
    S(const S &) = delete;
};

auto s_factory(){
    return S{}; // compiles in C++17, neither a copy nor a move.
}

#include <memory>

class Product {
public:
    virtual void use() const = 0;
};

class ConcreteProductA : public Product {
public:
    void use() const override {
        std::cout << "Using ConcreteProductA" << std::endl;
    }
};

class ConcreteProductB : public Product {
public:
    void use() const override {
        std::cout << "Using ConcreteProductB" << std::endl;
    }
};

class Factory {
public:
    std::unique_ptr<Product> createProduct(const std::string& productType) {
        if (productType == "A") {
            return std::make_unique<ConcreteProductA>();
        } else if (productType == "B") {
            return std::make_unique<ConcreteProductB>();
        } else {
            return nullptr;
        }
    }
};

int main() {
    Factory factory;
    
    std::unique_ptr<Product> productA = factory.createProduct("A");
    if (productA) {
        productA->use(); // Output: Using ConcreteProductA
    }
    
    std::unique_ptr<Product> productB = factory.createProduct("B");
    if (productB) {
        productB->use(); // Output: Using ConcreteProductB
    }
    
    return 0;
}

\end{verbatim}

Here, a factory class (`Factory`) that creates different types of products
(`ConcreteProductA` and `ConcreteProductB`) derived from an abstract base class `Product`.
The factory's `createProduct` method takes a product type as input and returns a unique pointer to the created product.
The main function creates instances of `ConcreteProductA` and `ConcreteProductB` through the factory.

\subsection{Virtual Factories}
A virtual factory is a design pattern that uses virtual functions and 
polymorphism to create objects of different types through a common interface.
Polymorphism is the ability of an object to take on many forms
and behave differently based on the context or the type of object it is being accessed through.

\begin{verbatim}
#include <memory>

class Product {
public:
    virtual void use() const = 0;
};

class ConcreteProductA : public Product {
public:
    void use() const override {
        std::cout << "Using ConcreteProductA" << std::endl;
    }
};

class ConcreteProductB : public Product {
public:
    void use() const override {
        std::cout << "Using ConcreteProductB" << std::endl;
    }
};

class AbstractFactory {
public:
    virtual std::unique_ptr<Product> createProduct() const = 0;
};

class ConcreteFactoryA : public AbstractFactory {
public:
    std::unique_ptr<Product> createProduct() const override {
        return std::make_unique<ConcreteProductA>();
    }
};

class ConcreteFactoryB : public AbstractFactory {
public:
    std::unique_ptr<Product> createProduct() const override {
        return std::make_unique<ConcreteProductB>();
    }
};

int main() {
    std::unique_ptr<AbstractFactory> factoryA = std::make_unique<ConcreteFactoryA>();
    std::unique_ptr<Product> productA = factoryA->createProduct();
    if (productA) {
        productA->use(); // Output: Using ConcreteProductA
    }
    
    std::unique_ptr<AbstractFactory> factoryB = std::make_unique<ConcreteFactoryB>();
    std::unique_ptr<Product> productB = factoryB->createProduct();
    if (productB) {
        productB->use(); // Output: Using ConcreteProductB
    }
    
    return 0;
}
\end{verbatim}

Here, an abstract factory class (`AbstractFactory`) that defines the interface for creating products.
The concrete factory classes (`ConcreteFactoryA` and `ConcreteFactoryB`) implement
implement the createProduct method to create specific products (ConcreteProductA and ConcreteProductB, respectively).
The main function demonstrates the usage of 
the virtual factory by creating instances of the concrete factories and using them to create products.

\chapter{Object Lifetime}

\section{Constructors}

\begin{verbatim}
City montreal;
montreal.population = 20000;

montreal.get_population();

class Song { // header file, song.hpp

  std::string title;

public:
  void add_title(std::string new_title);
  std::string get_title();
};

void Song::add_title(std::string new_title) {
  title = new_title; // main file, song.cpp
}

std::string Song::get_title() {
  return title;
}
int main() {
  Song electric_relaxation;
  electric_relaxation.add_title("Electric Relaxation");
  std::cout << electric_relaxation.get_title();
}
\end{verbatim}

\section{Destructors}

\begin{verbatim}
------ city.hpp -------

class City { 
  std::string name;
  int population;
 
public:
  City(std::string new_name, int new_pop);
  ~City();  // destructor
};
 
City::~City() { // in main, city.cpp
            // any final cleanup
}

class Song { // in header, .hpp
  std::string title;
  std::string artist;

public:
  Song(std::string new_title, std::string new_artist);
  ~Song();
  
  std::string get_title();   // getters
  std::string get_artist();
};

Song::Song(std::string new_title, std::string new_artist) // .cpp
  : title(new_title), artist(new_artist) {}

Song::~Song() {              // added destructor
  std::cout << "Goodbye " << title << "\n";
}

struct Double_Data {

    Double_Data(const std::size_t size)
      : data(new double[size] // allocate
    {
    }

    ~Double_Data() {          // destructor
      delete [] data;         // free
    }

    double *data;

};
\end{verbatim}

\subsection{Automatic Destructors}

\begin{verbatim}
    The object moves out of scope.
    The object is explicitly deleted.
    When the program ends.

void some_func() {
    std::string s{"Hello"};     // allocate a string
}                               // automatically frees it
                                // when scope exits
\end{verbatim}

\subsection{Deterministic Object Lifetime and Destruction}

Constructor / Destruct pairs (RAII) combined with scoped values give us determinism.

\begin{verbatim}
void some_func() {
    std::string s{"Hello"};     // allocate a string
}                               // automatically frees it
                                // when scope exits

Standard: c++ 98
\end{verbatim}


\subsection{RAII}

Constructor and Destructor pairs? I need more on this.

\subsection{New Keyword}

\begin{verbatim}
std::array<int, 3> ar{1,2,3};
int* dyn_ar = new int[4];
\end{verbatim}



\subsection{Automatic}
\begin{verbatim}

For format see p.44 filetime puzzler book 1

S object_1("a","t");
\end{verbatim}

\subsection{Thread\_local}
\begin{verbatim}
thread_local object_2("a","t");
\end{verbatim}

\subsection{Static}
\begin{verbatim}
static object_3("a","t");
\end{verbatim}

\subsection{Dynamic}

\begin{verbatim}
delete new S *object_4("a","t");
\end{verbatim}


\chapter{Engineering}

\section{Synch}
\section{Asynch}
\section{Multithreading - Parallelisation}

\section{Routine}
\section{Subroutine}


\subsection{Allocation Overhead}
\subsection{Ownership}
\subsection{Turing Complete}


\subsection{Guaranteed Copy Elision}

Compilers have "always done copy elision on return values but c++17 guarantees it in some situations:

\begin{verbatim}
struct S {
    S() = default;
    S(S&&) = delete;
    S(const S &) = delete;
};

auto s_factory(){
    return S{}; // compiles in C++17, neither a copy nor a move.
}
\end{verbatim}

\chapter{Timeline of Programs}

Where are sanitizers, compiler warnings, clang-tidy? In this timeline?

\section{Before Compilation}

\subsection{Before Compilation Tools}

\section{Compilation Time}

\subsection{Compile Time Tools}


\section{Runtime}

\subsection{Runtime Tools}


\chapter{Libraries}

A collection of pre-compiled code, providing functionality. Functions, classes and data structures,
that are not part of the standard C++ library.

\section{fmt - format library}

\begin{verbatim}
#include <format>
#include <string_view>

void print_map(const auto &map, const std::string_view &key_desc = "key",
                                const std::string_view &value_desc = "value")
{
    for (const auto &[key, value] : map) /// strucuted binding
    {
        std::puts(std::format("{}: '{}' {}: '{}'",
                         key_desc, key, value_desc, value).c_str());

        // this is genious spacing for readability
    }
}

Standard c++20
\end{verbatim}
\subsection{String\_View}

Observe string-like things without actually constructing a std::string!
A non-owning "view" of a string like structure.

\begin{verbatim}
#include <string_view>
std::string_view sv{some_string_like_thing}; // no copy

Standard c++17

These are passed-by-value on purpose.
String_view are cheap to copy. It is recommended to pass them by value.
The following code doesn't create a string anymore, if doesn't have to.

void print_map(const auto &map, const std::string_view &key_desc = "key",
                                const std::string_view &value_desc = "value")
{
    for (const auto &[key, value] : map) /// strucuted binding
    {
        std::cout << key_desc << ": '" << key << "' "
                  << value_desc << ": '" << value << "'\n";
    }
}

int main()
{
    print_map(get_some_map(), "index", "location");
}
\end{verbatim}

\subsection{Cout}

Std::Cout is quite verbose, relatively slow and difficult to reason about. 

\begin{verbatim}
Puts (algorithm)
\end{verbatim}

\subsection{Text Formatting}

A subset of the excellent {fmt} library, allowing for formatting of strings with positional,
named and python/printf style formatting options.

\begin{verbatim}
#include <format>
std::string s = fmt::format("I'd rathe be {1} than {0}.", "right", "happy");
// "I'd rather be happy than right."

Standard C++20
\end{verbatim}

\subsection{All Combined (algorithms, text formatting, concepts, etc.)}

\begin{verbatim}
#include <format>
#include <string_view>

void print_map(const auto &map, const std::string_view &key_desc = "key",
                                const std::string_view &value_desc = "value")
{
    const auto print_key_value = [&](const auto &data) { 
        const auto &[key, value] = data;
        std::puts(std::format("{}: '{}' {}: '{}'",
                         key_desc, key, value_desc, value).c_str());
    };

    for_each(begin(map), end(map), print_key_value);
}

// if only there was some way to not have to do this begin(map) and end(map)!!!

Auto Concept in c++20

#include <array>

auto get_data(const auto & ... params)
{
    return std::array{params...};
}
\end{verbatim}


\section{System()}

\begin{verbatim}
#include <string>

int main() {
    std::string cmd = "ls -l";
    system(cmd.c_str()); // change to c-string
    return 0;
}
\end{verbatim}

\subsection{Error Handeling - if cmd succeeded}

\begin{verbatim}
int main() {
    std::string cmd = "ls -l";
    int result = system(cmd.c_str());  // Pass C-string to system()

    if (result == 0) {
        std::cout << "Command executed successfully." << std::endl;
    } else {
        std::cout << "Command execution failed." << std::endl;
    }
    return 0;
}
\end{verbatim}

\section{Structured Bindings}

Used to decompose a structure or array into a set of identifiers.
You must use auto, and the number of elements must match. There's no way to skip an element.

\begin{verbatim}
const auto &[elem1, elem2] = some_thing;

Standard c++17

template<typename Map>
void print_map(const Map &map, const std::string &key_desc = "key",
                               const std::string &value_desc = "value")
{
    for (const auto &[key, value] : map) /// strucuted binding
    {
        std::cout << key_desc << ": '" << key << "' "
                  << value_desc << ": '" << value << "'\n";
    }
}

Standard c++17
\end{verbatim}

\subsection{Bulky Template Synthax}

\begin{verbatim}
#include <string>

template<typename Map>
void print_map(const Map &map, const std::string &key_desc = "key",
                               const std::string &value_desc = "value")
{
    for (const auto &[key, value] : map) /// strucuted binding
    {
        std::cout << key_desc << ": '" << key << "' "
                  << value_desc << ": '" << value << "'\n";
    }
}

Potential Inefficiency Hiding

void print_map(const auto &map, const std::string &key_desc = "key",
                                const std::string &value_desc = "value")
{
    for (const auto &[key, value] : map) /// strucuted binding
    {
        std::cout << key_desc << ": '" << key << "' "
                  << value_desc << ": '" << value << "'\n";
    }
}

int main()
{
    print_map(get_some_map(), "index", "location");
}

/// Std::String vs Const Char *

We are constructing a std::string from a const char * for no particular reason.

\end{verbatim}

\section{System Design}

\subsection{Header Files}

\subsection{Source Files}

\section{Arguments}

\begin{verbatim}
int main(int argc, char* argv[]) {
    return 0;
}
\end{verbatim}

\subsection{Flags}

\begin{verbatim}
bool hasFlag(const std::vector<std::string>& arguments, const std::string& flag) {
    return std::find(arguments.begin(), arguments.end(), flag) != arguments.end();
}

int main(int argc, char* argv[]) {
    std::vector<std::string> arguments(argv + 1, argv + argc);

    if (arguments.size() == 0) {
        arguments.push_back("-help");
    }

    if (hasFlag(arguments, "-o")) {
        std::cout << "Flag -o for -omit is present!" << std::endl;
    }
\end{verbatim}

\subsection{User Input}

\begin{verbatim}
std::cout << "Enter your password: ";
std::cin >> password;

#include <cstdlib>

int main() {
    std::string answer;
    std::cout << "Place the output in" << output_dir << "? [y/yes, n/no]: ";
    std::cin >> answer;

    if (answer == "y" || answer == "yes") {
    } else if (answer == "n" || answer == "no") {
    } else {
        std::cout << "Invalid response, exiting";
        std::exit(1) // (failure)
    }
    return 0;
}
\end{verbatim}



\section{Type}

\subsection{Return Types}

\begin{verbatim}
  std::exit(0) (success)
  std::exit(1) (failure)

  int isPrime(int x) {
    if (x % 2 == 1) {
        return 0; // is considered as false
                  // Yet, is it not the same as the boolean false
                  // In a boolean context, 
                  // 0 is implicitly converted to false,
                  // However, not equivalent in all contexts.
    }
  }

Void, Also refered as subroutine.

void oscar_wilde_quote() {
  std::cout << "The highest";
}
\end{verbatim}

\chapter{Filesystem Library}

LOOK AT FILESYSTEM VIDEO IN CODE

Introduced in c++17, the language has a filesystem library allowing
manipulation of paths, regular files and directories.

\section{Files}

\subsection{File Creation}

Use the fstream header and the ofstream function.

\begin{verbatim}
#include <filesystem>
#include <fstream> // Add this header for std::ofstream
namespace fs = std::filesystem;

int main() {
    fs::create_directories("sandbox/a");

    std::ofstream("sandbox/file.1.txt");     // is simplest, deletes instantly

    std::ofstream file("sandbox/file1.txt"); // initialize the variable on top!

    // Check if the file was opened successfully
    if (file.is_open()) {

        file << "This is some content written to the file.\n";

        file.close(); // Close the file after writing if no longer needed

    } else {
        std::cout << "Failed to open the file.\n";
    }

    return 0;
}
\end{verbatim}


\subsection{File Deletion}

\section{Paths}

\section{Directories}

\subsection{Directory Creation}

Create a directory with fs::create\_directories. This creates both sandbox and a, inside it.

\begin{verbatim}
#include <filesystem>
#include <fstream>
namespace fs = std::filesystem;

int main() {

    fs::create_directories("sandbox/a");
    std::ofstream("sandbox/file1.txt");
    std::ofstream("sandbox/file2.txt");

    for (auto& p : fs::directory_iterator("sandbox")) {
        std::cout << p.path() << '\n';
    } 

fs::remove_all("sandbox");

}
\end{verbatim}

\subsection{Directory Iterators}

\begin{verbatim}
#include <filesystem>
namespace fs = std::filesystem;

int main() {
    std::string path = "./foo";
    for (const auto &entry : fs::directory_iterator(path))
        std::cout << entry.path();
}
\end{verbatim}

\subsection{File Creation}

\subsection{Testing Framework}

\section{Benchmarking strategy}
\subsection{Chrono, clock time}

\begin{verbatim}
#include <chrono>
int main() {

  // Measure time taken for goodnight1():
  std::chrono::high_resolution_clock::time_point start = std::chrono::high_resolution_clock::now();

  std::cout << goodnight1("tulip");

  std::chrono::high_resolution_clock::time_point end = std::chrono::high_resolution_clock::now();
  std::chrono::duration<double, std::milli> time_span = end - start;

  // Print time taken for goodnight1():
  std::cout << "Time taken for goodnight1(): " << time_span.count() << " milliseconds.\n\n";
\end{verbatim}

\chapter{Good Practices}

\section{Proper Design}

\begin{verbatim}
    if a component is hard to test, it is not properly designed. 
    if a component is easy to test, it indicates proper design. 
    Approval tests ressource : https://cppcast.com/clare-macrae/ 
\end{verbatim}

\section{Warnings}

Enable as many compiler warnings as you can. 
Fix new warning generated. 

\textbf{It will feel tedious and meaningless}
\textbf{But this is the c++ way to catch real bugs.}

\section{Slow Down!}

\begin{verbatim}
Copy and pasting is easy. 
Forging ahead in comfort is too easy.
Plan ahead, don't get caught off guard. 
\end{verbatim}

\subsection{Ponder for solutions}

\begin{verbatim}
If the solution seems large or complex, stop. 
Walk and ponder for the solution. 
discuss the design with a rubber duck. 
spend less time programming, more thinking.
\end{verbatim}

\section{C++ is not magic nor Object-Oriented}

\begin{verbatim}
It's not magic, construct to test your doubts.
It is multi-disciplinary, supports all programming paradigms.

    Procedural
    Functional
    Object-Oriented
    Generic
    Compile-Time(contexpr and template metaprogramming)

Knowing when paradims are needed is the key to good C++.
Using appropriate techniques takes time and appropriate technique. 
\end{verbatim}

\subsection{Learn a different language}

\begin{verbatim}
    Lisp // Diverge from the C-family languages. Learn,
    Haskell
    Erlang
\end{verbatim}

\section{DRY (Don't Repeat Yourself)}

Principle aimed at reducing repetition of code and reusability. 
We have more tools today, concepts, generic lambdas, etc. to help with templates.

\begin{verbatim}
// divide doubles
double divide(double numerator, double denominator) {
    return numerator / denominator;
}

//But you don't want all of your divisions to be promoted to double. 

// Divide floats
float divide(float numerator, float denominator) {
    return numerator / denominator;
}

// You want to handle some kind of int values. 

//divide ints
int divide(int numerator, int denominator) {
    return numerator / denominator;
}

// Templates were designed for just this scenario. 
\end{verbatim}

\subsection{Auto Deduction Practice}

\begin{verbatim}
What is the type of val in all these situation?

const int *get();

int main() {
    const auto val = get();
}

const int &get();

int main() {
    const auto val = get();
}

The function get() returns a const int\& (a reference to a constant integer),
and the auto keyword deduces the type of val as the same as the type of the
expression on the right-hand side of the assignment.
Since get() returns a const int&, the type of val is also const int&.
The const qualifier in front of auto ensures that the type deduced for
val is also a const reference.

Const int *get();

int main() {
    const auto *val = get();
}

According to GPT

The function get() returns a const int* (a pointer to a constant integer),
and the auto keyword deduces the type of val as the same as the type of the
expression on the right-hand side of the assignment.
Since get() returns a const int*, the type of val is also const int*.

The const qualifier applies to the integer pointed to by val,
not to the pointer val itself. Therefore, val is a non-constant pointer to a constant integer.
If you wanted a constant pointer to a constant integer,
you would need to declare val as const int * const val = get();

const int &get();

int main() {
    const auto &val = get();
}

const int *get();

int main() {
    const auto &val = get();
}

const int &get();

int main() {
    const auto &&val = get();
}
\end{verbatim}

\chapter{Compilers}
\section{Compiler GNU GCC}

\begin{verbatim}
alias g++="g++ -Wall -Werror -Wextra -pedantic -std=c++17"
\end{verbatim}

\section{Compiler Clang}

\chapter{Program Timeline}


\section{Sanitizers (before compiling), link-time, compile-time, runtime).


